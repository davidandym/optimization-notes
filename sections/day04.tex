\subsection{Convex Functions}

\subsubsection{The Epigraph}

\begin{defn}{Convex Functions}{}
Let $S \subseteq \reals^n$ be a non-empty, convex set and $f: S \rightarrow \reals$
\medskip\\
$f$ is \textbf{convex} iff 
		$\forall 
		\vx, \vy \in S, \lambda \in [0, 1]$ it holds that
		\begin{frml}
		f\big(\lambda\vx + (1 - \lambda)\vy\big) \leq \lambda f(\vx)
		+ (1 - \lambda)f(\vy)
		\end{frml}
		\medskip\\
$f$ is \textbf{strictly convex} iff 
		$\forall 
		\vx, \vy \in S, \vx \neq \vy, \lambda \in (0, 1)$ it holds that
		\begin{frml}
		f(\lambda\vx + (1 - \lambda)\vy) < \lambda f(\vx)
		+ (1 - \lambda)f(\vy)
		\end{frml}

\end{defn}
In 2 dimensions
this is essentially just saying that, for any secant line between 2 
points, the function value has to be "under the secant line".

Note a few properties of convex functions:
\begin{itemize}
	\item
		$f$ is convex iff $-f$ is \textit{concave.}
	\item
		$\forall c \geq 0, cf$ is convex if $f$ is convex
	\item
		the sum of convex functions is convex
\end{itemize}

\begin{defn}{Epigraph}{}
Let $S \subseteq \reals^n$ be a convex, non-empty set, and 
$f: S \rarrw \reals$. The \textbf{epigraph} of $f$ is defined as
\begin{frml}
Ep(f) = \bigg\{ 
	\begin{bmatrix}
		\vx \\ \beta \\
	\end{bmatrix}
	\in \reals^{n + 1} : \vx \in S, \beta \in \reals, \beta \geq f(\vx)
\bigg\}
\end{frml}
\end{defn}

\textit{In 2 dimensions, you can sort of think of this as the area over the 
curve, or like the... air of a hilly 3-D landscape. 
It's everything above the function, if we added an extra dimension to
help visualize our space.}


\begin{theo}{}{}
Suppose $S \subseteq \reals^n$ is a convex, non-empty set
and $f : S \rarrw \reals$. Then $f$ is a convex function iff
$Ep(f)$ is a convex set.
\end{theo}

\textit{The proof of this is omitted, but is rather trivial with respect
to the definition of convex sets and functions.}

\subsubsection{The Support Theorem}

\begin{theo}{Support Theorem}{}

$S \subseteq \reals^n$ non-empty, convex set. Let $\bx \in Bndr(S)$. Then
$\exists$ a hyperplane containing $\bx \in \reals^n$ such that all of $S$ is
contained in one of the two half-spaces.
\end{theo}

\textit{Intuitively, we can think of this as follows: If a set of convex, then any of
it's boundary points define a tangent hyperplane which contains the boundary
point but \textbf{does not split the set}. In other words, the tangent hyperplane
\textbf{supports}, or holds the set up.}

\begin{theo}{}{}
Let $S \in \reals^n$ be a non-empty, open, convex-set, and let 
$f: S \rarrw \reals$ be continuously differentiable.
\medskip\\
Then
\begin{itemize}
	\item
		$f$ is convex iff $\forall \vx, \bx \in S , \;
		f(\vx) \geq f(\bx) + \nabla f(\bx)^T(\vx - \bx)$
	\item
		$f$ is strictly convex iff $\forall \vx, \bx \in S , \; \vx \neq \bx, \;
		f(\vx) > f(\bx) + \nabla f(\bx)^T(\vx - \bx)$
\end{itemize}
\end{theo}

\textit{This is basically showing us that the tangent hyperplane is the only possible
candidate for supporting hyperplane on the boundary of $Epi(f)$.}


\begin{theo}{}{}
Let $S \in \reals^n$ be a non-empty, open, convex set. Let 
$f: S \rarrw \reals$ be twice continuously differentiable. 
\medskip\\
Then $f$ is convex iff 
$\forall \vx \in S, \nabla^2 f(\vx)$ is \textit{Positive Semi-Definite}(PSD).
\end{theo}

\begin{proof}[]
($\impliedby$): Suppose $\nabla^2 f(x)$ is PSD for all points $\vx \in S$.
By Taylor $\forall \vx, \bx \in S$
\begin{frml}
	f(\vx) = f(\bx) + \nabla f(\bx)^T(\vx - \bx)
	+ \frac{1}{2} (\vx - \bx)^T \nabla^2f(\vz)(\vx - \bx)
\end{frml}
where $\vz$ is some point between $\vx, \bx$ also in $S$, since $S$ is convex.
Now, note that the last term is $\geq 0$, since the Hessian is PSD.
Thus,
\begin{frml}
	f(\vx) \geq f(\bx) + \nabla f(\bx)^T (\vx - \bx)
\end{frml}
and so $f$ is convex by Thm. 5.3.

($\implies$): Suppose $\exists \; \bx \in S$ s.t. $\nabla^2 f(\bx)$ is \textbf{not} PSD.
Then $\exists \; \vd \in \reals^n$ such that $\vd^T \nabla^2 f(\bx) \vd < 0$.
Then, by continuity of $\nabla^2 f$, there exists an epsilon neighborhood around 
$\bx$ such that $\forall \vz$ in this neighborhood, $\vd^T \nabla^2 f(\vz) \vd < 0$.
For any $\alpha > 0$ small enough such that $\vx = \bx + \alpha \vd$ in this
neighborhood, then by Taylor
\begin{frml}
	f(\vx) &= f(\bx) + \nabla f(\bx)^T (\vx - \bx) + \frac{1}{2}(\vx - \bx)^T
	\nabla^2 f(\vz) (\vx - \bx) \\
		   &= f(\bx) + \nabla f(\bx)^T (\vx - \bx) + \frac{1}{2}(\alpha \vd)^T
	\nabla^2 f(\vz) (\alpha \vd) \\
		   &= f(\bx) + \nabla f(\bx)^T (\vx - \bx) + \frac{1}{2}\alpha^2 \vd^T
	\nabla^2 f(\vz) \vd \\
\end{frml}

Note also that

\begin{frml}
	\alpha^2 \vd^T \nabla^2 f(\vz) \vd < 0 
	\implies f(\vx) < f(\bx) + \nabla f(\bx) (\vx - \bx)
\end{frml}
so $f$ is not convex, again by Thm 5.3.
\end{proof}

\begin{theo}{}{}
Let $S \in \reals^n$ be a non-empty, open, convex set. Let 
$f: S \rarrw \reals$ be twice continuously differentiable.
\medskip\\
Then, if $\forall \vx \in S, \; \nabla^2 f(\vx)$ is PD, then $f$ is 
\textbf{strictly convex}.
\end{theo}

\begin{proof}[]
Again by Taylor, $\forall \vx, \bx \in S$
\begin{frml}
	f(\vx) = f(\bx) + \nabla f(\bx)(\vx - \bx)
	+ \frac{1}{2}(\vx - \bx)^T\nabla^2 f(\vz)(\vz - \bx)
\end{frml}

As before, the last term here is strictly $> 0$, so $f(\vx) > f(\bx) + 
\nabla f(\bx) (\vx - \bx)$, which shows that $f$ is strictly convex.
\end{proof}

\textit{Note: $f$ strictly convex $\notimplies$ $\forall \vx \in S, \nabla^2 f(\vx)$ is 
PD. It \textit{does} imply that it's PSD everywhere.
To see a simple example, take $x^4$. This function is twice diff., and strictly
convex, but the second derivative at $x = 0$ is $0$.}


\begin{theo}{}{}
Let $S \subseteq \reals^n$ non-empty, convex, open, and $f: S \rarrw \reals$
be a convex, continuously differentiable function. Finally, let 
$P: min f(\vx), \; \vx \in S$.
\medskip\\
Then $\bx \in S$ is a global minimum of P iff $\bx$ is a local minimum iff it's a
stationary point. 
\medskip\\
\textit{(i.e. in a convex fn, if you are a local min, you are
a global min).}
\end{theo}

\begin{proof}[]
First note if $\bx$ is a global min, then it is a local min, which finally 
means it is a stationary point.

By convexity, and thm 5.3, 
$\forall \vx \in S$, $f(\vx) \geq f(\bx) + \nabla f(\bx) (\vx - \bx)$.
Since $\bx$ is a stationary point, $\nabla f(\bx) (\vx - \bx)$ is $0$, and thus
$\forall \vx \in S, \; f(\vx) \geq f(\bx)$ $\implies$ $\bx$ is a global minimum.

Finally, note that if $f$ is strictly convex, then the global min is strict as 
well.
\end{proof}

