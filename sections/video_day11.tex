\subsection{Barrier Methods}

Lastly, we will be considering \textit{Barrier Methods} of optimization,
which are sometimed also called ``\textit{interior penalty function method}''.
Similar to penalty methods, this will be a way of fitting our constrained
optimization problems into an unconstrained setting.
However, whereas in penalty methods we were largely considering points that were not
feasible, or \textit{exterior}, in barrier methods we will find our way through
feasible points, or \textit{interior points}, until we find ourselves on the
boundary of our region.

Consider the problem P:
\begin{frml}
	\min f(\vx) \st &\vg(\vx) \leq \vzero\\
					&\vx \in S
\end{frml}
where $S \subseteq \reals^n$ is non-empty, and all functions $f, g_i$ are
continuous.


\begin{defn}{Barrier Function}{}
	Define $\Lambda: (-\infty, 0) \rightarrow \reals$ be a continuous, non-negative
	function such that 
	\begin{frml}
		\lim_{t \rightarrow 0} \Lambda (t) = \infty
	\end{frml}
	A \textbf{barrier function}
		$B: \{ \vx \in S : \vg(\vx) < \vzero \} \rightarrow \reals$
	is defined by
	\begin{frml}
		B(\vx) = \sum_{i = 1}^m \Lambda(\vg_i(\vx))
	\end{frml}
\end{defn}

\textit{Note: As a $\vg_i(\vx)$ get's closer to 0, the value of $\Lambda(\vg_i(\vx))$
get's closer to $\infty$. As we get closer and closer to the boundaries of our
region, the barrier value goes up.}

Examples of $\Lambda$:
\begin{frml}
	B(\vx) = -\sum_{i = 1}^m \frac{1}{\vg(\vx)}
\end{frml}
and 
\begin{frml}
	B(\vx) = -\sum_{i = 1}^m ln(\min\{-\vg_i(\vx), 1\})
\end{frml}
\textit{Note: This last one, we can think of as having $0$ value for each constraint
whose value is} more negative \textit{than $-1$, and otherwise barrier values
start to occur.}

\begin{defn}{Barrier Auxiliary Function and Problem}{}
Given any barrier function $B$ and a scalar $\beta > 0$, the
\textbf{auxiliary function},  $\Psi_\beta: \{ \vx \in S : \vg(\vx) < \vzero \} \rightarrow \reals$,
is defined by 
\begin{frml}
	\Psi_\beta(\vx) = f(\vx) + \beta * B(\vx)
\end{frml}

\bigskip
The \textbf{auxiliary problem}, then is defined as $P_\beta$ :
\begin{frml}
	\min \Psi_\beta(\vx) \st &\vg(\vx) < \vzero \\
							 &\vx \in S
\end{frml}
\end{defn}

\textit{Note: In this formulation, we have made our feasible region something
of an open set. Now, we can utilize un-constrained optimization techniques
inside this set, since the Barrier Function will penalize us heavily for
approaching the tips of our feasible region. Then, as we dial down the $\beta$ 
parameter, we are allowed to approach the barrier}.



\begin{algorithm}\label{alg:barrier}
\caption{Barrier Function Algorithm}
	\KwIn{Barrier Function $B$, $\delta \st 0 < \delta < 1$, positive tolerance $\tau$}
	\KwOut{$\vx^*$ optimal in P}
	\For{$k = 1, 2, 3, \ldots$ until terminate}{
		Define auxiliary problem $P_{\delta^k}$ and solve (using unconstrained optimization techniques)\;
		$\bx \leftarrow$ solution to $P_{\delta^k}$ \;
		Terminate if $\delta^k * B(\vx)$ is less than $\tau$\;
	}
\end{algorithm}

This yields the \textbf{Barrier Function Algorithm}~(Algorithm \ref{alg:barrier}).





