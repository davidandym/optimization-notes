\subsection{The Simplex Method}
\label{sec:simplex-method}

\subsubsection{The Simplex Tableau}

Consider an LP in standard form, where $\vA \in \reals^{m \times n}$ is rank  $m$:
\begin{frml}
	\min \vc^T\vx \st &\vA\vx = \vb, \\ &\vx \geq \vzero
\end{frml}

WLOG, $\vA = \mat{B & | & N}$. where $B = \reals^{m\times m}$ invertible. \textit{Since
$A$ is rank $m$, we can row-reduce to make this true}.

\begin{defn}{Pre-Tableau}{}
All notation as above.

\medskip
The \textbf{pre-tableau} is the \textit{augmented matrix} defined as:

\begin{frml}
	PreT = \mat{1 & -\vc^T & | & 0 \\ \vzero & \vA & | & \vb} =
	\mat{1 & -\vc^T & | & 0 \\ \vzero & B\;\; |\;\; N & | & \vb} 
\end{frml}

\end{defn}

In essence, we can think of this augmented matrix as implying that there is some
slack variable $z \in \reals \st z - \vc^T\vx = 0$ (on the top row), and
asserting that $A\vx = \vb$ in all the other rows.

However, we are actually interested in \textit{Basic Feasible Tableaus} in the simplex method.
To convert this into \textit{basic form}, we can use the invertible matrix 
$H \in \reals^{m+1 \times m+1}$:

\begin{frml}
	H = \mat{1 & \vc_B^TB^{-1} \\ \vzero & B^{-1}}
\end{frml}

Which allows us to convert out Pre-tableau into a \textit{basic tableau}:

\begin{defn}{Basic Tableau and Basic Feasible Tableau}{}
All notation as above.

\medskip
A \textbf{Basic Tableau} is any tableau of the form:
\begin{frml}
	BasicTableau = H \times PreT = 
	\mat{1 &  \vzero^T  & -\vc_N^T + \vc_B^TB^{-1}N & | & \vc_B^TB^{-1}\vb \\
	\vzero & I & B^{-1}N & | & B^{-1}\vb}
\end{frml}

If $B^{-1}\vb \geq \vzero$, then $BasicTableau$ is called the \textbf{Basic
Feasible Tableau} of basis $B$ with basic feasible solution $\bx = \mat{\bx_B \\ 0}$.
In this case, we can rewrite the Tableau as:

\begin{frml}
	\mat{1 & \vzero^T & -\vr_N^T & | & \textbf{ofv}(\bx) \\
	\vzero & I & \ldots & | & \bx_B}
\end{frml}
\end{defn}

Firsly, note that if $B^{-1}\vb \geq \vzero$, then $\bx = \mat{B^{-1}\vb \\ \vzero}$
is feasible in LP, making it a basic feasible solution and thus making the
tableau a \textit{basic feasible tableau}).

From any basic tableau, we can derive a couple key things. We can write out
$z$, our ``slack variable'' for the objective function value in terms of
the non-basic components of $\vx$:
\begin{frml}
	&z - (\vc_N^T - \vc_B^TB^{-1}N)\vx_N = \vc_B^TB^{-1}\vb \\
	\implies &z = \vc_B^TB^{-1}\vb + (\vc_N^T - \vc_B^TB^{-1}N)\vx_N
\end{frml}
which (if $B^{-1}\vb \geq \vzero$) is \textit{exactly} the 
$\textbf{ofv}(\vx)$ written our with respect to the \textbf{ofv} of the basic
feasible solution $\bx$ and the reduced cost coefficients:
\begin{frml}
	z = \textbf{ofv}(\bx) + r_N^T\vx_N
\end{frml}
We can similarly write out the basic components of $\vx$ using the non-basic
components:
\begin{frml}
	\vx_B = B^{-1}\vb - B^{-1}N\vx_N = \bx_B - B^{-1}N\vx_N 
\end{frml}

Lastly, note that if the top row (not the scalars on either end) is $\leq \vzero$
in a basic feasible tableau,
then our bfs is optimal 
(because they represent the negated reduced cost coefficients).

\subsubsection{Pivoting and the Simplex Method}

The \textbf{Simplex Method} is a way to move from one Basic Feasible Tableau
to another Basic Feasible Tableau with a few guarantees. 
One guarantee is that the next BFT is, in fact, a basic feasible
tableau as well. The other guarantee is that,
as we iterate, the objective function value of the bfs for the basic feasible
tableau is always going down or staying constant. 
Thus, the simplex method is a way of iterating through
basic feasible solutions of an $(LP)$, via their respective BFTs, until we find
the optimal BFS.

Let's assume we start with some basic feasible tableau, representing a basic feasible 
solution.

\begin{frml}
	InitBFT = \mat{1  & -\hat{\vc}^T & | & z \\ \vzero & \hat{\vA} & | & \hat{\vb}}
\end{frml}

\begin{algorithm}\label{alg:simplex-method}
\caption{The Simplex Method}
	\KwIn{$InitBFT$}
	\KwOut{The optmal bfs, $\bx$}
	$CurrBFT \leftarrow InitBFT$\;
	\While{Top row $-\hat \vc^T$ is not all negative}{
		Pick any non-basic column (non-identity column), $\hat A^{(j)}$,
		with top row value  $-\hat \vc_j > 0$\;
		\If{$\hat A^{(j)} \leq \vzero$}{
			The problem is unbounded.$^{\diamondsuit}$ Exit\;
		}
		$i^* \leftarrow \argmin_{i \in \hat A_{i,j} > 0} 
		\frac{\hat \vb_i}{\hat A_{i,j}}$, (i.e. pick from the rows whose value
		$\hat A_{i,j}$ is positive)\;
		Row reduce $CurrBFT$ to convert column $\hat A^{(j)}$ into the
		indentity column with a value of $\hat \vc_j = 0$ above it$^{\triangle}$\;
		$CurrBFT \leftarrow $ new, row-reduced $BFT$\;
	}
\end{algorithm}

$\diamondsuit$:
		Why? $\vx_B = \hat{\vb} - \vx_j\hat{\vA}^{(j)} \geq \vzero$ as
		$\vx_j \rarrw \infty$. So, we can keep reducing our $\vx_j$ forever,
		and we'll always have a feasible solution. Additionally, because the
		$-\hat{\vc}^T_j \geq 0 \implies \vr^T_j \leq 0$, then our ofv is either
		staying the same of decreasing as $\vx_j \rarrw \infty$.

$\triangle$ : This new tableau is still a basic tableau! We still have the same
identity property, we've just changed one of the identity columns.

\begin{theo}{}{}
The result of one iterate of the simplex method is a basic feasible tableau with
ofv going down.

\medskip
\textit{I.e. the new $CurrBFT$ is a basic feasible tableau whose basic feasible
solution, $\bx$, has $\textbf{ofv}(\bx)$ lower than the previous iterate's
bfs.}
\end{theo}

\begin{proof}[]

The previous column with identity where $i^*=1$ will no longer be an identity 
column.  Any other columns with 
identities will not be touched by the row operations (because those
columns have $0$ for rows $i^*$. So the property of containing an 
identity still holds for our tableau, and we can say the same for the
top row. Thus, we still have a basic tableau.


$\forall i \neq i^*$, we now have that our new $\hat{\vb}_i = 
\hat{\vb}_i - \frac{\hat{\vA}_{i,j}}{\hat{\vA}_{i^*,j}}\hat{\vb_{i^*}}$.
Note that in this term, we know $\hat{\vb}_i, \hat{\vb}_{i^*} \geq 0$
and $\hat{\vA}_{i^*,j} > 0$. Thus, if $\hat{\vA}_{i,j} \leq 0$, then we
know that our new $\hat{\vb}_i > 0$. If $\hat{\vA}_{i,j} > 0$, then we
know that $\frac{\hat{\vb}_i}{\hat{\vA}_{i,j}} > 
\frac{\hat{\vb}_{i^*}}{\hat{\vA}_{i^*,j}}$ and thus 
$\hat{\vb}_i \geq 0$. Therefore, the basic feasible solution for our
new basis is feasible, 

Lastly, the new $\hat{z} = \hat{z} + 
\frac{\hat{\vc}_j}{\hat{\vA}_{i^*,j}}\hat{\vb}^{i^*}$. Note that
$\hat{\vc}_j < 0$ (that's why we picked it), $\hat{\vA}_{i^*,j} > 0$
(again, that's why we picked it), and $\hat{\vb}_{i^*} \geq 0$ (by
feasibility). This will always decrease if $\hat{\vb}_{i^*} > 0$,
otherwise we will stall (which is typically okay).
\end{proof}

\textit{Note: If $\hat{b}_{i^*} = 0$, then your ofv will not go down for that 
iteration. This is called \textbf{degeneracy} or \textbf{stalling}. It is not
uncommon to experience stalling while running this algorithm. However, 
\textbf{cycling} \textit{is} uncommon, and rarely happens.}

\begin{theo}{}{}
If your $(LP)$ is feasible, and the ofv has a lower bound, then the simplex 
method will terminate, and $(LP)$ has a solution at the bfs represented by the
final BFT.
\end{theo}

\subsubsection{Big-M method}

How do we pick the first basis in the algorithm?
If we are not given a basis, we can create one.
Let the original problem, P, be:
\begin{frml}
	\min \vc^T\vx \\
	\st \vA\vx = \vb, \vx \geq \vzero
\end{frml}
We can augment P to create an artificial basis. 
Choose $M >> 0$. Write $P^*$ as:
\begin{frml}
	\min \mat{\vc \\ \mathbf{M}}^T \mat{\vx \\ \vw} 
	\st &\mat{\vA & | & I} \mat{\vx \\ \vw} = \vb, \\&\mat{\vx \\ \vw} \geq \vzero
\end{frml}

Note that all the feasible solutions for the original LP are contained within
the feasible solutions for this new LP, since they are obtained by setting
$\vw = \vzero$. Since $M >> 0$, it will never be optimal to have a solution in
this new $(LP)^*$ where $\vw \neq \vzero$, and thus we should recover a solution
that works equally as well for our original LP.
However, importantly, our augmented $\vA$ matrix already contains an identity
in it, which constitutes as our fist basis!
