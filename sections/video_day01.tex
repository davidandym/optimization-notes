\pagebreak
\section{Moving beyond linear problems}

Most sections from now on will largely 
\textit{not} make the assumption that the problems are linear.

\subsection{Quadratic Programs}

Another set of problems that we will be considering often are \textit{Quadratic
Programs}. Quadratic programs are often of the following form:

\begin{defn}{Quadratic Program}{}
	Let $Q \in \reals^{n \times n}, \vc \in \reals^n, \vx \in \reals^n$, and
	$S \subseteq \reals^n$.

	\medskip
	A \textbf{Quadratic Program}, QP, is of the form:
	\begin{frml}
		\min \vx^TQ\vx + \vc^T\vx \st \vx \in S
	\end{frml}
\end{defn}

Note that the gradient and Hessian of this function $f(\vx) = \vx^TQ\vx + \vc^T\vx$ are: 

\begin{frml}
	\nabla f(\vx) &= Q\vx + \vc \\
	\nabla^2 f(\vx) &= Q
\end{frml}

In general, we are going to consider problems where $Q$ is positive definite or
positive semi-definite. If a single eigenvalue of $Q$ is negative, then this 
problem is, in general, NP-Hard and thus very difficult to solve. Therefore, 
when we consider quadratic problems, we will often be considering convex or 
strictly convex quadratics.

\subsubsection{Existence and Uniqueness of solutions}

Let's consider a less general quadratic program. Let $Q \in \reals^{n\times n}$ be
symmetric, positive definite. Let $\vc \in \reals^n, \; \vx \in \reals^n$.
Finally let $\mathcal{S}\subseteq \reals^n$ be a \textit{closed}, non-empty set.
Consider the problem, QP:
\begin{frml}
	\min \vx^TQ\vx + \vc^T\vx \st \vx \in \mathcal{S}
\end{frml}

\begin{prop}{}{}
	All notation as above.

	\medskip
	There $\exists$ a global minimizer of QP. 

	Furthermore, if $\mathcal{S}$ is also convex, then the minimizer is unique.
\end{prop}

\textit{Note: This last statement cannot be taken for granted. For example, the
set $\mathcal{S} = \{ \vx : f(\vx) = 35\}$ is a closed, non-empty set. However,
every point of $\mathcal{S}$ is a global minima!}.

\textit{Note: This problem definition covers a lot of problems that we're interested in.
For example, if $\mathcal{S}$ is a polyhedral set, then $\mathcal{S}$ is closed and
convex! Or if $\mathcal{S}$ is defined by a set of functions $g_i \leq 0$ 
where all $g_i$ are convex and continuous then $\mathcal{S}$ is again a closed,
convex set.}

\begin{proof}[]
	Define $f(\vx) = \vx^TQ\vx + \vc^T\vx$. For any  $\vd \in \reals^n \st ||\vd||_2 = 1$,
	and for any $\alpha \in \reals^n_{\geq 0}$ let $\bx = \alpha*\vd$.
	\begin{frml}
		f(\bx) &= \bx^TQ\bx + \vc^T\bx \\
			   &= \alpha^2\vd^TQ\vd + \alpha\vc^T\vd \\
			   &\geq^{\diamondsuit} \bigg(\min_{\lambda \in \sigma(Q)} \lambda \bigg)\alpha^2
			   - ||\vc||_2 \alpha
	\end{frml}

	\textit{$\diamondsuit$ : The first term of the inequality is a special property
	of symmetric matrices and unit vectors. This value is bounded by the max
and min eigenvalues of $Q$, which in this case are always $> 0$, because Q
is P.D. The second term is due to the fact that $| \vc^T\vd | \leq ||\vc||*||\vd||=||\vc||$
and thus the minimum value of  $\vc^T\vd = -||\vc||$.}

Note that this bound scales only by $\alpha$. All other terms are constant.
Additionally, the bound is dominated by $\alpha^2 * \min \lambda$, which is
positive.
Thus, for any arbitrary $\vx_0 \in \mathcal{S}$, $\exists \alpha > 0 \st 
\forall ||\vx||_2 > \alpha \; f(\vx) > f(\vx_0)$. Then define 
\[B = \{\vx \in \reals^n : ||\vx||_2 \leq \alpha\}\]
Note that is $B$ is closed and thus $B \cap \mathcal{S}$ is closed. $B$ is bounded,
and thus $B \cap \mathcal{S}$ is bounded. Therefore, $B \cap \mathcal{S}$ is a
compact, non-empty set. We know that it's non-empty because, at the very least,
$\vx_0$ is in it!

Because $B$ is compact and non-empty, we know that there must exist a global min
of $f$ on $B$, $\vx^*$. Therefore,  $\vx^*$ is a global min of $f$ on $\mathcal{S}$,
because $\forall \vx \notin B, f(\vx) > f(\vx_0) \geq f(\vx^*)$. Thus a global
min exists for  $f$ on $\mathcal{S}$.

Finally, assume $\mathcal{S}$ is a convex set. Suppose $\exists$ two global mins,
$\vx'$ and $\vx''$ for $f$ on $\mathcal{S}$. Let 
\[D = \{ \vx \in \reals^n : f(\vx) \leq f(\vx')\}\]
Note that $D \cap \mathcal{S}$ is also a convex set, and thus the line segment
$[\vx', \vx''] \in D \cap \mathcal{S}$. Thus $f$ is constant along the line segment,
which is a constradiction as $f$ is strictly convex. Therefore, the global min
of $f$ on $\mathcal{S}$ is unique.

\end{proof}
