
% \title{Introduction to Problem Setting}
% \author{Non-Linear Optimization - 553.762}
% \date{}
% 
% \begin{document}

% \maketitle
\pagebreak
\section{Preliminaries}

We start off by introducing some formal notation and fundamental theorems in topology.
This is going to be a lot of definitions, so I apologize in advance for that.

\subsection{Vectors}

\begin{defn}{Magnitude}{}
	The \textbf{magnitude} of a vector is, intuitively, what we can think of as the length
	of that vector from the origin. It's defined as, $\forall \vx \in \reals^n$:
\begin{frml}
	||\vx|| = ||\vx||_2 = (\sum_{i=0}^{n} \vx_i^2)^{\frac{1}{2}}
\end{frml}
\end{defn}

\begin{defn}{Distance}{}
	The \textbf{distance} between two vectors $\vx, \vy$ is defined as $\forall \vx, \vy \in \reals^n$:
	\begin{frml}
		dist(\vx, \vy) = || x - y ||
	\end{frml}
\end{defn}

\begin{defn}{Epsilon-Neighborhood}{}
The \textbf{epsilon-neighborhood} of a vector $\vy\in\reals^n$ is defined $\forall \vx\in\reals^n,
\epsilon>0$ as the set: 

$$N_\epsilon (\vy) = \{ \vx \in \reals^n : dist(\vx, \vy) < \epsilon \}$$
\end{defn}

\subsection{Sets}

Let $S \subseteq \reals^n$

\begin{defn}{Interior and Boundary Point}{intboundpoint}

	$\forall \vx \in \reals^n, \vx$ is an \textbf{interior point} of S, 
		$\vx \in Int(S) \iff \exists \epsilon > 0 \st N_{\epsilon}(\vx) \subseteq S$

\bigskip
$\forall \vx \in \reals^n, \vx$ is a \textbf{boundary point} of S, $\vx \in Bndr(S)$, if
$\forall \epsilon > 0$, $ N_{\epsilon}(\vx)$ contains a point in $S$ and a point in $S^C$.
\end{defn}

In other words, if there exists some small enough epsilon neighborhood of $\vx$ 
which is entirely contained within $S$, then $\vx \in Int(S)$. The opposite
of this is a boundary point, that is
for \textbf{all} epsilon neighborhood around $\vx$, $\vx$ contains both a point
in $S$ and a point outside of $S$. 
\textit{Note that a point cannot be an interior point
and a boundary point simultaneously, by definition.}

\begin{defn}{Open and Closed Set}{openclosedset}
$S$ is \textbf{open} if $\forall \vx \in S, \vx \in Int(S)$.
That is, if $S = Int(S)$. 

\bigskip
$S$ is \textbf{closed} if all boundary points of $S$ are \textbf{in} $S$,
i.e. if $Bndr(S) \subseteq S$. 
\end{defn}

To see some intuitive examples of open and closed sets, we can imagine subsets of
a numberline
\begin{itemize}
\item The range [2,3] is closed in $\reals$, but not open.
\item The range (2,3) is open, but not closed.
\item The range [2, 3) is neither open nor closed.
\item And finally, if we trivially consider $S = \reals$, then $S$ is both
	open and closed, since it has no boundary points.
\end{itemize}

\begin{defn}{Bounded Set}{boundedset}
$S$ is \textbf{bounded} if $\exists \epsilon > 0$ such that 
$S \subseteq N_{\epsilon} ( \vzero )$.
In other words, if it's sort of centered around the origin by some epsilon.

\textit{We can imagine this as saying that our set }
\end{defn}

\begin{defn}{Compact Set}{compactset}
$S$ is \textbf{compact} if for any sequence 
$\{ \vx^i \}_{i=1}^{\infty} \subseteq S$, then there exists a convergent 
subsequence $\{\vx^{i,j}\}_{j=1}^{\infty}$, and $\vy \in S$ such that
$\vx^{i,j} \rightarrow \vy$.
\end{defn}

\begin{prop}{}{}
For any $S \in \reals^n$, $S$ is open if and only if $S^C$ is closed.
\end{prop}

\begin{proof}[Proposition 1.1]
Trivially, $Bndry(S) = Bndry(S^C)$.
Also, each point in $S$ is either $Int(S)$ or $Bndry(S)$ but not both.
Finally, $S$ is open iff $S = Int(S)$ which is true iff $Bndry(S) \subseteq S^C$
which is true iff $Bndry(S^C) \subseteq S^C$, i.e. $S^C$ is closed.
\end{proof}

\begin{theo}{}{}
$\forall S \in \reals^n$, nonempty, $S$ is compact iff $S$ is
both closed and bounded.
\end{theo}

\begin{theo}{}{}
If $S \in \reals^n$ is compact, and you have $f: S \rightarrow \reals$ continuous, 
then $\exists \bar{\vx} \in S$ such that
$\forall \vx \in S, f(\bar{x}) \leq f(\vx)$ and $f(\bar{\bar{\vx}}) \geq f(\vx)$.
\end{theo}

\subsection{Problem Setting}

We will typically work with problems of the following form:
Suppose $S \subseteq \reals^n$, $f: S \rightarrow \reals$.
Define the problem P as: 
\begin{frml}
	\min \;&f(\vx) \st \vx \in S
\end{frml}

In this setting we call:
\begin{itemize}
	\item
		$f$ the objective function
	\item
		$S$ the feasible region
	\item
		$\vx$ the decision variables
\end{itemize}
The set $S$ may contain equation constraints, and if $\vx \in S$ then we say that
$\vx$ is feasible.
Some definitions:

\begin{defn}{Local and Global Minimum}{}
If $\bar{\vx} \in S$ satisfies $\exists \epsilon > 0$ s.t. 
$\forall \vx \in N_{\epsilon}(\bar{\vx}) \cap S$, $f(\bar{\vx}) \leq f(\vx)$, then
$\bar{\vx}$ is a \textbf{local minimum}.

If $\bar{\vx} \in S$ satisfies $\exists \epsilon > 0$ s.t. 
$\forall \vx \in N_{\epsilon}(\bar{\vx}) \cap S$, $\vx \neq \bx$, $f(\bar{\vx}) < f(\vx)$, then
$\bar{\vx}$ is a \textbf{strict local minimum}.

\bigskip

If $\bar{\vx} \in S$ satisfies $\forall \vx \in S, f(\bar{\vx}) \leq f(\vx)$ 
then we call $\bx$ the \textbf{global minimum}, 

If $\bar{\vx} \in S$ satisfies 
$\forall \vx \in S, \vx \neq \bar{\vx}, f(\bar{\vx}) < f(\vx)$, 
then $\bar{\vx}$ is a \textbf{strict global minimum}.

\bigskip
If $\bx$ is a global min of $f$, then we call $f(\bar{\vx})$ the 
\textbf{optimal objective function value (oofv)}.
\end{defn}

If $\forall \epsilon > 0, \exists \vx \in S$, such that $f(\vx) < -\epsilon$,
then P is unbounded, and has \textit{no global min}.
