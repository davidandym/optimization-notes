\subsection{The Lagrangian Dual}

Now, let's take a look at what the \textit{dual} form of this looks like, where I
get to pick $\lambda$ first:
We can define a similar function as before:
\begin{defn}{Lagrangian $\Theta$-function and Lagrangian Dual Problem}{}
Where, $\forall \lambda \in \reals^m_{\geq 0}$ we define:
\begin{frml}
\Theta(\lambda) = \inf_{\vx \in S} \mathcal{L}(\vx, \lambda)
\end{frml}
and we define the Lagrangian dual problem (DP):
\begin{frml}
	\sup_{\lambda\in\reals^m_{\geq 0}} \Theta(\lambda) = \sup_{\lambda \in\reals^m_{\geq 0}} \inf_{\vx} \mathcal{L}(\vx, \lambda)
\end{frml}
\end{defn}

\begin{theo}{Weak Duality}{}
	If $\vx_*$ feasible in P, $\lambda_*$ feasible in DP, then 
	$\textbf{ofv}_{DP}(\lambda_*) \leq \textbf{ofv}_{LP} (\vx_*)$
\end{theo}

\begin{proof}[]
	Notice that
	\begin{frml}
	\Theta(\lambda_*) = \inf_{x \in S} \mathcal{L}(\vx, \lambda_*) \leq \mathcal{L}(\vx_*, \lambda_*)
	\leq \sup_{\lambda\in\reals^m_{\geq 0}} \mathcal{L}(\vx_*, \lambda) = \Phi(\vx_*)
	\end{frml}

	We can see by similar arguments as we used in the game theory arguments earlier.
	In essence, $\mathcal{L}(\vx_*, \lambda_*)$ is a value which has a fixed
	$\vx_*$ and a fixed $\vlambda_*$ (\textit{neither of which is necessarily optimal}). 
	This is obviously less than or greater than
	the $\sup$ or $\inf$ if we allow the problem to consider \textit{all} $\lambda \in \reals^m_{\geq 0}$ or
	\textit{all} $\vx \in S$ or 
\end{proof}

\begin{theo}{Supervisor Principle}{}
	If $\vx_*$ feasible in P, $\lambda_*$ feasible in DP
	$\st \textbf{ofv}_{DP}(\lambda_*) = \textbf{ofv}_{P}(\vx_*)$ then
	$\vx_*$ optimal in P and $\lambda_*$ optimal in DP.
\end{theo}

For classes of problems where this happens, we say there is \textit{no duality-gap}.
If we have strict inequality ($<$) then we say that there\textit{ is a duality-gap}.

\subsection{Example: Linear Pogram}

Let's look at an example where this duality can occur: 
Suppose $A \in \reals^{m\times n}, \vb \in \reals^m,
\vc \in \reals^n$ such that:
\begin{frml}
	P: \min \vc^T\vx \st A\vx\geq \vb, \vx \geq \vzero
\end{frml}

\subsubsection{A slightly unfair formulation}

We want to compute the Lagrangian dual. We can phrase this as 
$\vb - A\vx \leq \vzero = \vg(\vx)$ and $\vx \geq \vzero = \vx \in S$.

\textit{This is actually \textit{not a fair translation} of the problem, since we
are packing some of our restrictions (some $\vg$ functions) into
restrictions on $S$ instead (making it a closed set). 
There is a better way to do this, which we'll see in a second.}

We now have the Lagrangian function: 
\begin{frml}
\forall \vx \geq \vzero, \lambda \in \reals^m_{\geq 0}, \;\;
	\mathcal{L}(\vx, \lambda) &= \vc^T\vx + \lambda^T(\vb - A\vx) \\
							   &= \lambda^T\vb + (\vc^T - \lambda^TA)\vx
\end{frml}

What does the dual problem look like for this Lagrangian? Remember in the dual problem,
I am choosing lambdas and trying to \textit{maximize} the overall function value $\sup \Theta(\lambda)$.
Since $\lambda \geq \vzero$, let's see what our options are for $\lambda$
\begin{itemize}
	\item if $\lambda^TA \leq \vc^T$ then $(\vc^T - \lambda^TA) \geq \vzero$, and since $\vx \geq \vzero$, then
		$\Theta(\lambda) = \inf_{\vx \geq \vzero} \mathcal{L}(\vx, \lambda) = \lambda^T\vb$ since \textit{you}
		will always choose $\vx = \vzero$ to minimize the function value.
	\item if $\lambda^TA \not \leq \vc^T$ then $(\vc^T - \lambda^TA) \not \leq \vzero$ and
		now you can choose some component of $\vx$ to be $\infty$ which will make the
		whole value of the function go to $-\infty$. 
		So, $\Theta(\lambda) = \inf_{\vx \geq \vzero} \mathcal{L}(\vx, \lambda) = -\infty$
\end{itemize}
So, essentially, the dual of the Lagrangian can be re-written as:
\begin{frml}
	\sup_{\vlambda \geq \vzero} \Theta(\vlambda) = \max \vlambda^T\vb \st &\vlambda^TA \leq \vc,\; \vlambda \geq \vzero
\end{frml}
which is \textit{exactly} the Dual Formulation of the original P (a linear program).

\subsubsection{A slightly cleaner formulation}

Now, let's write this our more fairly.
\begin{frml}
	P: \min \vc^T\vx \st &\mat{\vb \\ \vzero} + \mat{-A \\ -I}\vx \leq \vzero,
							\vx \in \reals^n
\end{frml}

Now, our $S$ is an open set ($\reals^n$) and all of our constraints are packed into
the $\vg$ function, and this is a nicer way to write this out.

To write out the Lagrangian now, let's write out our variables as $\lambda = \mat{\vy \in \reals^m \\ \vz \in \reals^n}$
and now our Lagrangian looks like,
$\forall \vx \in \reals, \vy \in \reals^m \geq \vzero, \vz \in \reals^n \geq \vzero$:

\begin{frml}
\mathcal{L}(\vx, \lambda) = \vc^T\vx + \lambda^T \bigg( \mat{\vb \\ \vzero} + \mat{-A \\ -I}\vx \bigg)
= \vc^T\vx + \vy^T(\vb - A\vx) + \vz^T(-\vx) = \vy^T\vb + (\vc^T - \vy^TA - \vz^T)\vx
\end{frml}

As above,let's break up our possible lambda values:
\begin{itemize}
	\item if $\vc^T - \vy^TA - \vz = \vzero$ then $\Theta(\vlambda) = \inf_{\vx \in \reals} \mathcal{L}(\vx, \vlambda) = \vy^T\vb$
	\item if $\vc^T - \vy^TA - \vz \neq \vzero$ then  $\Theta(\vlambda) = \inf_{\vx \in \reals} \mathcal{L}(\vx, \vlambda) = -\infty$ 
		\textit{(remember in this case $\vx$ is not restricted to be positive).}
\end{itemize}
And so the dual ultimately looks like
\begin{frml}
	\max \vb^T\vy \st \vc^T - \vy^TA = \vz, \; \vy,\vz \geq \vzero
\end{frml}
which we can simplify, since, in this formulation, $\vz$ only appears once and is only
restricted to be positive, as:
\begin{frml}
	\max \vb^T\vy \st \vc^T - \vy^TA \geq \vzero, \; \vy \geq \vzero
\end{frml}
which we \textit{finally} can simplify one more time to look like:
\begin{frml}
	\max \vb^T\vy \st A^T\vy \leq \vc, \; \vy \geq \vzero
\end{frml}
which is a very comfortable form of the dual of the LP :) 

So we've seen that
converting our linear programming problem into it's Lagrangian form and solving the Lagrangian dual
is the same as solving the dual of the LP.

\subsection{Example: A quadratic program}

Let's look at another example:
Suppose $A \in \reals^{n\times n}$ sym. PD, $\vb \in \reals^n$ non-zero, $\vc \in \reals_{> 0}$ and
\begin{frml}
	P: \min \frac{1}{2}\vx^TA\vx \st \vb^T\vx + \vc \leq \vzero
\end{frml}
and we want to compute and solve the dual.
Well, the lagrangian looks like
\begin{frml}
	\forall \vx \in \reals^n, \lambda \geq \vzero, \; \mathcal{L}(\vx, \lambda)
	= \frac{1}{2}\vx^TA\vx + \lambda(\vb^T\vx + c)
\end{frml}
Now, note that 
\begin{frml}
	\Theta(\lambda) = \inf_{\vx \in \reals^n} \frac{1}{2}\vx^TA\vx + \lambda(\vb^T\vx + c)
\end{frml}

this function is actually convex with respect to $\vx$. So once we pass in a
given $\lambda$, the minimization will just solve for the $\vx$ which is a global
min of that function. And, since $A$ is invertible (it's PD) that means that we can
directly solve for it numerically. For a given $\lambda$, the $\vx$ that minimizes
the langrangian is given as $\vx = -\lambda A^{-1}\vb$

We can just plug this into our original $\Theta$ function, then, to get
\begin{frml}
	\Theta(\lambda) &= \frac{1}{2}(-\lambda A^{-1}\vb)^TA(-\lambda A^{-1}\vb) + 
\lambda(\vb^T(-\lambda A^{-1}\vb) + c) \\
					 &= \frac{\lambda^2}{2}\vb^TA^{-1}\vb - \lambda^2\vb^TA^{-1}\vb + \lambda c
\end{frml}
and thus the (DP) looks like
\begin{frml}
	\sup_{\lambda \geq 0} \frac{-\lambda^2}{2}\vb^TA^{-1}\vb + \lambda c
\end{frml}

if we set the derivative $-\lambda \vb^TA^{-1}\vb + c = 0 \rarrw \lambda = \frac{c}{\vb^TA^{-1}\vb} > 0$
which is a unique global max, making the \textbf{ofv} to be $\frac{c^2}{2\vb^TA^{-1}\vb}$.
So, in this case, we have used the Lagrangian dual to exactly solve our
system! It's a pretty powerful tool!

