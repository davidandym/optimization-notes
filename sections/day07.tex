\pagebreak
\section{The Dual of Linear Programs}

We have seen Linear Programs and how to solve them. Now we will look at the
\textit{dual problem} of a Linear Program.

\begin{defn}{Dual of the Canonical Form}{}
Let $\vA \in \reals^{m \times n}, \; \vb \in \reals^m, \; \vc \in \reals^n$.
A linear program in canonical form is of the form:
		\begin{frml}
			\min \vc^T\vx \st &\vA\vx\geq\vb, \\ &\vx \geq \vzero
		\end{frml}

Then the \textbf{dual problem}, DP, takes the form:
\begin{frml}
\max \vb^T\vy \st &\vA^T\vy \leq \vc, \\ &\vy \geq \vzero
\end{frml}
\end{defn}

\begin{defn}{Dual of the Standard Form}{}
Let $\vA \in \reals^{m \times n}, \; \vb \in \reals^m, \; \vc \in \reals^n$.
A linear program in standard form is of the form:
\begin{frml}
\min \vc^T\vx \st &\vA\vx=\vb, \\ &\vx \geq \vzero
\end{frml}

Then the \textbf{dual problem}, DP, takes the form:
\begin{frml}
\max \vb^T\vy \st \vA^T\vy \leq \vc
\end{frml}
\end{defn}

\subsubsection{Equality across forms}

As before, with the primary Linear Programming formulation, the 2 forms 
(canonical and dual) are equivalent to each other.


Suppose we have a linear program, LP, in \textit{canonical form}:
\begin{frml}
	\min \vc^T\vx, \st &\vA\vx \geq \vb, \\ &\vx \geq \vzero \\
\end{frml}
It can be converted into a \textit{standard form LP}: 
\begin{frml}
	\min \mat{\vc \\ \vzero} \mat{\vx \\ \vz}, \st &\mat{\vA & | & -I}
	\mat{\vx \\ \vz} = \vb, \\ &\mat{\vx \\ \vz} \geq \vzero \\
\end{frml}
With \textit{standard form} dual problem, DP: 
\begin{frml}
	\max \vb^T\vy \st \mat{\vA^T \\ -I}\vy \leq \mat{\vc \\ \vzero} \\
\end{frml}
Which is exactly the following DP in \textit{canonical form}:
\begin{frml}
	\max \vb^T\vy \st &\vA^T\vy \leq \vc, \\ &\vy \geq \vzero
\end{frml}

The other direction is similar: 
Suppose we are given a linear program, LP, in \textit{standard form}:
\begin{frml}
	\min \vc^T\vx \st \vA\vx = \vb, \; \vx \geq \vzero
\end{frml}
The \textit{canonical form} of this LP is:
\begin{frml}
	\min \vc^T\vx \st &\mat{\vA \\ -\vA}\vx \geq \mat{\vb \\ -\vb}, \\ &\vx 
	\geq \vzero \\
\end{frml}
The dual of this problem is:
\begin{frml}
	\max \mat{\vb \\ -\vb}^T \mat{\vu \\ \vv} \st &\mat{\vA^T & | -\vA^T}
	\mat{\vu \\ \vv} \leq \vc, \\ &\mat{\vu \\ \vv} \geq \vzero \\
\end{frml}
Which can be converted into \textit{standard dual form} as:
\begin{frml}
	\max \vb^T(\vu - \vv) \st &\vA^T(\vu - \vv) \leq \vc, \\ &\mat{\vu \\ \vv}
	\geq \vzero 
\end{frml}
We can simplify this by setting $\vz = (\vu - \vv)$:

\textit{Note: that the non-negativity
constraints disappear for  $\vz$ because it is the difference of two  
non-negative vectors}
\begin{frml}
	\max \vb^T\vz \st &\vA^T\vz \leq \vc
\end{frml}


\subsubsection{The dual of the dual}

Lastly, note that we can show the dual of the dual is the primal:
\begin{frml}
	(LP): &\min \vc^T\vx, \; \st \vA\vx \geq \vb, \; \vx \geq \vzero &\rarrw \\
	(Dual): &\max \vb^T\vy, \st \vA^T\vy \leq \vc, \; \vy \geq \vzero &\rarrw \\
	(Equivalently):&\min -\vb^T\vy \st -\vA^T\vy \geq -\vc, \; \vy \geq \vzero &\rarrw \\
	(Dual): &\max -\vc^T\vz \st [-\vA^T]^T\vz \geq -\vb, \; \vz \geq \vzero &\rarrw \\
	(Resolving): &\min \vc^T\vz \st \vA\vz \geq \vb. \; \vz \geq \vzero
\end{frml}
The last term here is exactly our original LP.


\subsection{Duality}

\begin{theo}{Weak Duality}{}
Let LP be a lienar program, with a dual DP.	

\medskip
If $\bx$ feasible in LP, and $\by$ feasible
in DP, then $$\textbf{ofv}_{DP}(\by) \leq \textbf{ofv}_{LP}(\bx)$$
\end{theo}

\begin{proof}[Weak Duality (standard form)]
	\[\vb^T\by = (\vA\bx)^T\by = 
	\bx^T\vA^T\by \leq \bx^T\vc = \vc^T\bx\]
		The inequality holds by the feasability of $\by$ in the dual, which 
		states that $\vA^T\vy \leq \vc$.
\end{proof}

\begin{proof}[Weak Duality (canonical form)]
	\[\vb^T\by \leq (\vA\bx)^T\vy = 
	\bx^T\vA^T\by \leq \bx^T\vc = \vc^T\bx\]
\end{proof}

\begin{theo}{Supervisor Principle}{}
Let LP be a lienar program, with a dual DP.	

\medskip
If $\bx$ feasible in LP, $\by$ feasible
in DP, and \[\textbf{ofv}_{DP} (\by) = \textbf{ofv}_{LP} (\bx)\] then $\bx$ and $\by$
are optimal in their respective problems.
\end{theo}

\begin{proof}[]
By weak duality.
\end{proof}

\begin{theo}{Strong Duality}{}
Let LP be a linear program.

\medskip
If LP is feasible, and the objective 
function value is lower bounded then:
\begin{itemize}
	\item The LP has a solution $\bx$
	\item The DP has a solution $\by$
	\item $\textbf{ofv}_{DP}(\by)=\textbf{ofv}_{LP}(\bx)$.
\end{itemize}
\end{theo}

\begin{proof}[]
Let $\vA \in \reals^{m \times n}$ (WLOG assume it is rank $m$),
and $\vb \in \reals^m, \; \vc \in \reals^n$. Let LP be a linear program in
standard form:
\begin{frml}
	\min \vc^T\vx \st &\vA^T\vx = \vb, \\ &\vx \geq \vzero
\end{frml}
Assume that LP has been solved to optimality with the simplex method, with the
optimal bfs $\bx$ as the solution, and basis
$B$, implying
$A = \mat{B & | & N}$.
By optimality, $\vr_N^T = \vc_N^T - \vc_B^TB^{-1}N \geq \vzero$.
The dual formulation of this problem is:
\begin{frml}
	\max \; \vy^T\vb \st &\vy^T\vA \geq \vc^T
\end{frml}
Consider $\by^T = \vc_B^TB^{-1}$. Note that 
$\by^T\vb = (\vc_B^TB^{-1})\vb = \vc_B^T(B^{-1}\vb) = \vc^T\bx$. Thus, if
$\by$ is feasible, then it has $\textbf{ofv}_{DP}(\by) = \textbf{ofv}_{LP}(\bx)$ 
and by the Superisor Principle, $\by$ must be optimal.
If $\by\vA \leq \vc$ then $\by$ is feasible. This holds, because
\begin{frml}
	\by^T\vA = \vc_B^TB^{-1}\mat{B & | & N} = \mat{\vc_B^T & | & \vc_B^TB^{-1}N}
	\leq \mat{\vc_B^T & | & \vc_N^T}
\end{frml}

\textit{Note: The inequality holds because the reduced cost variables show 
that $\vc_N^T \geq \vc_B^TB^{-1}N$.}

Thus $\by$ is feasible with \textbf{ofv} equal to $\textbf{ofv}_{LP}(\bx)$ 
in the primal, and thus is optimal in DP.
\end{proof}

\subsubsection{Complementary Slackness}

\begin{defn}{Complementarity}{}

\begin{frml}
	\vx = \mat{63 \\ 0 \\ 0 \\ 9 \\ 0 \\ 2}, \; \vy = \mat{0 \\ 0 \\ 8 \\ 0 \\ 10 \\ 0}
\end{frml}

are \textbf{complementary}, since the non-zero entries of one correspond to zero
entries in the other.

If $\vx, \vy \in \reals^n \st \vx, \; \vy \geq \vzero$, then $\vx$ and $\vy$
are complementary iff $\vx^T\vy = 0$.
\end{defn}

\begin{theo}{Standard Form Complementary Slackness}{}
Let LP be a linear program in standard form, with dual problem DP.

\medskip
If $\bx$ feasible in LP,
$\by$ feasible in $DP$, then $\bx$ is optimal in the primal, $\by$ is optimal
in the dual iff $\bx$ is complementary to the dual slack of $\by$, i.e.
\begin{frml}
	\bx_{\geq \vzero} \perp (\vc - \vA^T\by)_{\geq \vzero}
\end{frml}
\end{theo}

\begin{proof}[]
As in the weak duality proof, 
\begin{frml}\vb^T\by = \bx\vA\by \leq \bx^T\vc = \vc^T\bx
\end{frml}
Note that $\bx$ is optimal in LP and $\by$ is optimal in DP iff $\vb^T\by = \vc^T\bx$.
which is true iff the equality $\bx^T\vA\by = \bx^T\vc$ holds, which is true
iff $\bx^T(\vc - \vA^T\by) = 0$!
\end{proof}

\begin{theo}{Canonical Form Complementary Slackness}{}
Let LP be a linear program in canonical form, with dual problem DP.

\medskip
If $\bx$ feasible
in LP, $\by$ feasible in DP, $\bx$ is optimal in LP and $\by$ is optimal in DP 
iff
$\bx \perp \vc - \vA^T\by$ \textbf{and} $\by \perp \vA\bx - \vb$.
\end{theo}

\begin{proof}[] 
\begin{frml}
	\vb^T\by \leq (\vA\bx)^T\by = \bx^T\vA^T\by \leq \bx^T\vc = \vc^T\bx
\end{frml}
Similar to above, we need our inequalities to become equalities, which happens
exactly when
\begin{itemize}
	\item
		$\vb^T\by = (\vA\bx)^T\vy \rarrw (\vA\bx - \vb)^T\by = 0$, and
	\item
		$\bx^T\vA^T\by = \bx^T\vc \rarrw \bx^T(\vc - \vA^T\by) = 0$
\end{itemize}
\end{proof}
