\subsection{Deriving the KKT Conditions}

We will derive the KKT conditions by making a slow march through
a series of propositions.

Let's start by considering the problem P, defined as:
\begin{frml}
	\min f(x) \st  &g_1(x) \leq 0, \\
						&g_2(x) \leq 0, \\
						&\vdots \\
						&g_m(x) \leq 0, \\
						&\vx \in S
\end{frml}

where we write $\vg(\vx) = \mat{g_1(\vx) \\ g_2(\vx) \\ \vdots \\ g_m(\vx)} 
\st \vg: \reals^n \rarrw \reals^m$, with constraints $\vg(\vx) \leq \vzero$.

\begin{defn}{Active Constraints}{}
For any $\vx \in S$, feasible in $P$,
the \textbf{active constaints} are 
\[
	\mathcal{A}_\vx = \{ i : g_i(\vx) = 0\}
\]
\textit{In other words, they are the constraints for which $\vx$ is on the ``boundary''.}
\end{defn}

\begin{prop}{KKT Prop 1}{}
If $\vx \in S$, feasible (satisfies $\vg$), and is a \textit{local min}
of $P$, then there does not exist a direction $\vd \in \reals^n$ which
is simultaneously a descent direction for $f$ and all active constraints, $g_i
\st i \in \mathcal{A}_\vx$.
\end{prop}

\begin{proof}[]
If such a $\vd$ exists then $\exists \; \alpha > 0$ small enough such that
\begin{frml}
	\forall i \in \mathcal{A}_\vx, g_i(\vx + \alpha\vd) < g_i(\vx) = 0  \\
	\forall i \notin \mathcal{A}_\vx, g_i(\vx + \alpha\vd) 
	\approx g_i(\vx) < 0 \\
\end{frml}
thus $\vx + \alpha\vd$ is feasible, and $f(\vx + \alpha\vd) < f(\vx)$
and thus $\vx$ is not a local min.
\end{proof}

\begin{prop}{KKT Prop 2}{}
If $\vx \in S$ is feasible and a local min of $P$ then
$\centernot \exists \vd \in \reals^n$ (direction, $\neq \vzero$), s.t.
$J^T\vd < 0$, where $J^T = \mat{\nabla f^T(x) \\ 
\nabla \vg_{\mathcal{A}_\vx}^T(\vx)}$
\end{prop}

\begin{proof}[]
Immediate consequence of prop 1, $J$ represents the gradients of $f$
and all active constrants $\mathcal{A}_\vx$, and
if such a $d$ exists, then it is a descent direction for both $f$ and all 
active constraints $\mathcal{A}_\vx$.
\end{proof}

\begin{prop}{KKT Prop 3}{}
If $\vx \in S$, feasible, is a local min of $P$, then
$\exists \vlambda \geq \vzero, \vlambda \neq \vzero, \st J\vlambda = \vzero$
\end{prop}

\begin{proof}[]
This is equivalent to Prop 2, by Gordon's Theorem.
\end{proof}

\begin{prop}{KKT Prop 4}{}
If $\exists \vx \in S$, feasible, is a local min of $P$,
then $\exists \mat{\beta \\ \vlambda} \in \reals^{1 + m}$, non-zero, such that
\begin{frml}
\mat{\nabla f(\vx)  & \nabla g_1(\vx)  & \ldots  & \nabla g_m(\vx)}
\mat{\beta \\ \vlambda} = \vzero, \st \forall i = 1, \ldots, m \; \vlambda_i 
g_i(\vx) = 0
\end{frml}
\end{prop}

Note that this is just a restating of Prop 3 where we include \textit{all} 
constraints here (not just active ones), but
if a constraint is not active (say $j$ is not active), then $\vlambda_j$ must
be zero.

Fritz-John is a re-statement of Prop 4, but utilizing the definition of the
\textit{Jacobian}:

\begin{defn}{The Jacobian}{}
The \textbf{Jacobian} of $\vg$ is defined as
\begin{frml}
	\nabla \vg(\vx) = \mat{\nabla g_1(\vx) & | & \nabla g_2(\vx) & | & \ldots
										   & | & \nabla g_m(\vx)}
\end{frml}
\end{defn}

Using the above definition and notes, we can rewrite Prop 4 as the Fritz-John
Optimality Conditions:

\begin{theo}{Fritz-John Optimality Conditions}{}
If $\vx \in S$, feasible in
$P$, is a local min of $P$, then $\exists \vlambda \in \reals^m, \beta \in \reals \st$
\begin{frml}
	&\beta \nabla f(\vx) + \nabla \vg(\vx) \vlambda = \vzero, \\
	&\vlambda^T\vg(x) = 0, \\
	&\text{and } \mat{\beta \\ \vlambda} \geq \vzero, \neq \vzero \\
\end{frml}
Which we can similarly write as...
\begin{frml}
	&\beta \nabla f(\vx) + \sum_{i=1}^m \nabla \vg_i(\vx) \vlambda_i = \vzero, \\
	&\sum_{i=1}^m \vlambda_i\vg_i(x) = 0, \\
	&\text{and } \beta, \vlambda_1, \vlambda_2, \ldots, \vlambda_m \geq 0, 
	\text{but not all $0$}
\end{frml}
\end{theo}

Again, there is no proof. This is just a re-statement of proposition 4, using the
definition of a Jacobian.
Finally, this leads us to the KKT conditions...

\begin{theo}{Karush-Kuhn-Tucker Conditions}{}
If $\vx \in S$, feasible in
$P$ is a local min of $P$ and satisfies the "constraint qualifications" (cq) that
$\{\nabla g_i\}_{i \in \mathcal{A}_{\vx}}$ are linearly independent, then
\begin{frml}
	\exists \vlambda \in \reals^m \st &\nabla f(\vx) + \nabla \vg(\vx)\vlambda
	= \vzero, \\
									  &\vlambda^T\vg(x) = 0, \\
									  &\text{and } \vlambda \geq \vzero
\end{frml}
\end{theo}

\begin{proof}[]
This follows from Fritz-John. The linear independence of $\{\nabla \vg_i \}_{i \in
\mathcal{A}_\vx}$ forbids $\beta$ from being $0$. Thus, we can divide
$\mat{\beta \\ \vlambda} / \beta$ which sets $\beta = 1$, and thus we only
require that $\vlambda \geq \vzero$ (since $\beta > 0$, the "but not all zero"
constraint is already satisfied). The $\vlambda_i$ are called the KKT 
multipliers.
\end{proof}
