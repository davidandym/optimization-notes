\subsection{Duality and the Simplex Tableau}

Recall the Simplex Method and Simplex Tableau from Linear Programs 
(\S~\ref{sec:simplex-method}).
The same construction can be used in solving the dual problem, as well!
Let LP be the following linear program:
\begin{frml}
	\min \vc^T\vx \st &\vA\vx = \vb,\\ &\vx \geq \vzero
\end{frml}
Let $B$ be a basis of LP, with the basic tableau:
\begin{frml}
	\mat{1 &  \vzero & \ldots & | & \vc_B^TB^{-1}\vb \\
		\vzero  & I & \ldots & | & B^{-1}\vb}
\end{frml}
and basic solution $\bx = \mat{
B^{-1}\vb \\ \vzero}$. Recall, additionally, that $\bx$ is feasible iff
$\bx\geq\vzero$. 
Set 
$\by^T = \vc_B^TB^{-1} \in \reals^m$. Note that $\by$ is not necessarily 
feasible in the dual problem, either.
However, our choices of $\bx$ and $\by$ 
vectors have the same objective function value (by design). 
This means that a basic tableau for which both $\bx$ and $\by$ are feasible 
has found the optimal solution for both LP and DP in $\bx$ and $\by$, respectively.

The items needed to a basic tableau to be optimal, with vectors $\bx$ and $\by$,
is:
\begin{enumerate}
	\item
		Feasability of $\bx$:
		\begin{enumerate}
			\item $\vA\vx = \vb$
			\item $\bx \geq \vzero$.
		\end{enumerate}
	\item Feasibility of $\by$ : $\by^T\vA \leq \vc^T$
	\item Equivalent \textbf{ofv}s: $\vc^T\bx = \vb^T\by$
\end{enumerate}

Notice that 1(a) is \textit{always true!}. If $B$ is a valid basis of $A$,
then it holds that $\mat{ B & | & N } \mat{B^{-1}\vb \\ \vzero}
= \vb$. Thus, the choice of $\bx$ satisfies this constraint always.
Additionally, by definition of $\by$ and $\bx$, 4 is always true as
well.

\textbf{Thus, solving the LP comes down to finding a basic tableau which satisfies 
(1b) and (2).}
(1b) is true precisely when $B^{-1}\vb \geq \vzero$.
(2) is true precisely when the top row of the simplex is 
$\leq \vzero$. To see this, first notice that the top row is equal to 
$-\vc^T + \vc_B^TB^{-1}\vA = -\vc^T + \by^T\vA$. (You can verify this 
easily, by just reasoning through the values that are on the top row). Thus,
the top row is actually the \textit{negative dual slack}
$\vc^T - \by^T\vA$. If this slack is $\geq \vzero$, then $\by$ is feasible.
Thus if the top row of the tableau is $\leq \vzero$, then $\by$ is 
feasible in DP.

\textit{Note: this also gives us a second way to think about the top row - it is 
both the rate of change of the ofv as you vary the non-basic variables, and it
is the negative of the dual slack.}

This gives us 2 paradigms to operate in when solving an LP.
\begin{enumerate}
	\item
		If we have a basic tableau which is primal feasible, but not dual 
		feasible, then we can use the primal simplex method to move towards
		a basis which has dual feasibility, while maintaining primal 
		feasibility.
	\item
		If we have a basic tableau which is dual feasible, but not primal
		feasible, then we can use the \textbf{dual simplex method} to move
		towards a basis that is primal feasible, while maintaining dual 
		feasibility.
\end{enumerate}

\subsubsection{Dual Simplex Method}

The \textit{dual simplex method} (Algorithm~\ref{alg:dual-simplex-method}) is very similar to the primal simplex method.
Given any basic dual feasible tableau ($-\hat{\vc}^T \leq \vzero$)
\begin{frml}
BDFT =	\mat{1 & | & -\hat{\vc}^T & | & \hat{z} \\
	\vzero & | & \hat{\vA} & | & \hat{\vb} }
\end{frml}

\begin{algorithm}\label{alg:dual-simplex-method}
\caption{The Dual Simplex Method}
	\KwIn{$InitBDFT$}
	\KwOut{The optmal bfs, $\bx$}
	$CurrBDFT \leftarrow InitBDFT$\;
	\While{$\hat \vb$ is not $\geq\vzero$}{
		$i^* \leftarrow $ some row $i \st \hat\vb_i < 0$ \;
		\If{Row $i^*$ of $\hat A,\;\;\hat A_{:,i^*} \geq 0$}{
			The the LP is infeasible.$^{\diamondsuit}$ Exit\;
		}
		$j^* \leftarrow \argmin_{j \in \hat A_{i^*,j} < 0} 
		\frac{- \hat \vc_j}{\hat A_{i^*,j}}$, (i.e. pick from the columns, $j$, 
		whose value $\hat A_{i^*,j}$ is negative)\;
		$CurrBDFT \leftarrow $ Pivot on $(i^*, j^*)$ (similar to simplex method),
		to obtain a new basic dual feasible tableau with an identity column in
		column $j^*$\;
	}
\end{algorithm}

$\diamondsuit$ : Recall that $\hat\vb_{i^*} < 0$. However, if $\hat A_{:,i^*} \geq 0$
then $\vb$  \textit{cannot} be $< 0$. Thus the problem is not feasible.


